\documentclass[]{article}
\usepackage{lmodern}
\usepackage{amssymb,amsmath}
\usepackage{ifxetex,ifluatex}
\usepackage{fixltx2e} % provides \textsubscript
\ifnum 0\ifxetex 1\fi\ifluatex 1\fi=0 % if pdftex
  \usepackage[T1]{fontenc}
  \usepackage[utf8]{inputenc}
\else % if luatex or xelatex
  \ifxetex
    \usepackage{mathspec}
  \else
    \usepackage{fontspec}
  \fi
  \defaultfontfeatures{Ligatures=TeX,Scale=MatchLowercase}
\fi
% use upquote if available, for straight quotes in verbatim environments
\IfFileExists{upquote.sty}{\usepackage{upquote}}{}
% use microtype if available
\IfFileExists{microtype.sty}{%
\usepackage{microtype}
\UseMicrotypeSet[protrusion]{basicmath} % disable protrusion for tt fonts
}{}
\usepackage[margin=1in]{geometry}
\usepackage{hyperref}
\hypersetup{unicode=true,
            pdftitle={ANP Brasil},
            pdfauthor={Luciano Teixeira},
            pdfborder={0 0 0},
            breaklinks=true}
\urlstyle{same}  % don't use monospace font for urls
\usepackage{color}
\usepackage{fancyvrb}
\newcommand{\VerbBar}{|}
\newcommand{\VERB}{\Verb[commandchars=\\\{\}]}
\DefineVerbatimEnvironment{Highlighting}{Verbatim}{commandchars=\\\{\}}
% Add ',fontsize=\small' for more characters per line
\usepackage{framed}
\definecolor{shadecolor}{RGB}{248,248,248}
\newenvironment{Shaded}{\begin{snugshade}}{\end{snugshade}}
\newcommand{\KeywordTok}[1]{\textcolor[rgb]{0.13,0.29,0.53}{\textbf{#1}}}
\newcommand{\DataTypeTok}[1]{\textcolor[rgb]{0.13,0.29,0.53}{#1}}
\newcommand{\DecValTok}[1]{\textcolor[rgb]{0.00,0.00,0.81}{#1}}
\newcommand{\BaseNTok}[1]{\textcolor[rgb]{0.00,0.00,0.81}{#1}}
\newcommand{\FloatTok}[1]{\textcolor[rgb]{0.00,0.00,0.81}{#1}}
\newcommand{\ConstantTok}[1]{\textcolor[rgb]{0.00,0.00,0.00}{#1}}
\newcommand{\CharTok}[1]{\textcolor[rgb]{0.31,0.60,0.02}{#1}}
\newcommand{\SpecialCharTok}[1]{\textcolor[rgb]{0.00,0.00,0.00}{#1}}
\newcommand{\StringTok}[1]{\textcolor[rgb]{0.31,0.60,0.02}{#1}}
\newcommand{\VerbatimStringTok}[1]{\textcolor[rgb]{0.31,0.60,0.02}{#1}}
\newcommand{\SpecialStringTok}[1]{\textcolor[rgb]{0.31,0.60,0.02}{#1}}
\newcommand{\ImportTok}[1]{#1}
\newcommand{\CommentTok}[1]{\textcolor[rgb]{0.56,0.35,0.01}{\textit{#1}}}
\newcommand{\DocumentationTok}[1]{\textcolor[rgb]{0.56,0.35,0.01}{\textbf{\textit{#1}}}}
\newcommand{\AnnotationTok}[1]{\textcolor[rgb]{0.56,0.35,0.01}{\textbf{\textit{#1}}}}
\newcommand{\CommentVarTok}[1]{\textcolor[rgb]{0.56,0.35,0.01}{\textbf{\textit{#1}}}}
\newcommand{\OtherTok}[1]{\textcolor[rgb]{0.56,0.35,0.01}{#1}}
\newcommand{\FunctionTok}[1]{\textcolor[rgb]{0.00,0.00,0.00}{#1}}
\newcommand{\VariableTok}[1]{\textcolor[rgb]{0.00,0.00,0.00}{#1}}
\newcommand{\ControlFlowTok}[1]{\textcolor[rgb]{0.13,0.29,0.53}{\textbf{#1}}}
\newcommand{\OperatorTok}[1]{\textcolor[rgb]{0.81,0.36,0.00}{\textbf{#1}}}
\newcommand{\BuiltInTok}[1]{#1}
\newcommand{\ExtensionTok}[1]{#1}
\newcommand{\PreprocessorTok}[1]{\textcolor[rgb]{0.56,0.35,0.01}{\textit{#1}}}
\newcommand{\AttributeTok}[1]{\textcolor[rgb]{0.77,0.63,0.00}{#1}}
\newcommand{\RegionMarkerTok}[1]{#1}
\newcommand{\InformationTok}[1]{\textcolor[rgb]{0.56,0.35,0.01}{\textbf{\textit{#1}}}}
\newcommand{\WarningTok}[1]{\textcolor[rgb]{0.56,0.35,0.01}{\textbf{\textit{#1}}}}
\newcommand{\AlertTok}[1]{\textcolor[rgb]{0.94,0.16,0.16}{#1}}
\newcommand{\ErrorTok}[1]{\textcolor[rgb]{0.64,0.00,0.00}{\textbf{#1}}}
\newcommand{\NormalTok}[1]{#1}
\usepackage{graphicx,grffile}
\makeatletter
\def\maxwidth{\ifdim\Gin@nat@width>\linewidth\linewidth\else\Gin@nat@width\fi}
\def\maxheight{\ifdim\Gin@nat@height>\textheight\textheight\else\Gin@nat@height\fi}
\makeatother
% Scale images if necessary, so that they will not overflow the page
% margins by default, and it is still possible to overwrite the defaults
% using explicit options in \includegraphics[width, height, ...]{}
\setkeys{Gin}{width=\maxwidth,height=\maxheight,keepaspectratio}
\IfFileExists{parskip.sty}{%
\usepackage{parskip}
}{% else
\setlength{\parindent}{0pt}
\setlength{\parskip}{6pt plus 2pt minus 1pt}
}
\setlength{\emergencystretch}{3em}  % prevent overfull lines
\providecommand{\tightlist}{%
  \setlength{\itemsep}{0pt}\setlength{\parskip}{0pt}}
\setcounter{secnumdepth}{0}
% Redefines (sub)paragraphs to behave more like sections
\ifx\paragraph\undefined\else
\let\oldparagraph\paragraph
\renewcommand{\paragraph}[1]{\oldparagraph{#1}\mbox{}}
\fi
\ifx\subparagraph\undefined\else
\let\oldsubparagraph\subparagraph
\renewcommand{\subparagraph}[1]{\oldsubparagraph{#1}\mbox{}}
\fi

%%% Use protect on footnotes to avoid problems with footnotes in titles
\let\rmarkdownfootnote\footnote%
\def\footnote{\protect\rmarkdownfootnote}

%%% Change title format to be more compact
\usepackage{titling}

% Create subtitle command for use in maketitle
\newcommand{\subtitle}[1]{
  \posttitle{
    \begin{center}\large#1\end{center}
    }
}

\setlength{\droptitle}{-2em}

  \title{ANP Brasil}
    \pretitle{\vspace{\droptitle}\centering\huge}
  \posttitle{\par}
  \subtitle{Evolução do Preço dos combustíveis em 2012}
  \author{Luciano Teixeira}
    \preauthor{\centering\large\emph}
  \postauthor{\par}
      \predate{\centering\large\emph}
  \postdate{\par}
    \date{11 de setembro de 2018}


\begin{document}
\maketitle

\section{Comandos R de preparação e transformação de
dados}\label{comandos-r-de-preparacao-e-transformacao-de-dados}

Resolva os exercícios a seguir utilizando os comandos no software
RStudio através do RMarkdown. Gere um relatório em .doc com os
comandos/código e as respectivas saídas e seus comentários.

\begin{itemize}
\tightlist
\item
  Exercício 1 - Crie um dataframe com 10 colunas e 40 linhas e imprima
  na tela as primeiras 6 linhas;
\item
  Exercício 2 - Mostre a que classe pertence cada uma das 10 colunas do
  seu dataframe;
\item
  Exercício 3 - Utilize um comando que faça uma avaliação exploratória
  das variáveis do seu dataframe e imprima na tela;
\item
  Exercício 4 - Adicione mais uma coluna gerada a partir da
  transformação de uma das 10 originais;
\item
  Exercício 5 - Faça um gráÒco que mostre a distribuição dos dados dessa
  nova variável do exercício 4.
\end{itemize}

\section{Dados da ANP - Agência Nacional do
Petróleo}\label{dados-da-anp---agencia-nacional-do-petroleo}

\subsection{Evolução de Preço dos Combustíveis no último Bimestre de
2012}\label{evolucao-de-preco-dos-combustiveis-no-ultimo-bimestre-de-2012}

\subsubsection{Resolução dos Exercícios}\label{resolucao-dos-exercicios}

1.Crie um dataframe com 10 colunas e 40+ linhas

Carregando a biblioteca \texttt{readr} para leitura do arquivo em csv

\begin{Shaded}
\begin{Highlighting}[]
\KeywordTok{library}\NormalTok{(readr)}
\end{Highlighting}
\end{Shaded}

Definindo a diretório de trabalho com a função \texttt{setwd()}.

\begin{Shaded}
\begin{Highlighting}[]
\KeywordTok{setwd}\NormalTok{(}\StringTok{"~/GitHub/GeneralRepositoriesUnisinos/PosUnisinosIntroducaoPythonR/Tarefa"}\NormalTok{)}
\end{Highlighting}
\end{Shaded}

Lendo o arquivo ``brasil.csv''.

\begin{Shaded}
\begin{Highlighting}[]
\NormalTok{anpbrasil <-}\StringTok{ }\KeywordTok{read_delim}\NormalTok{(}\StringTok{"brasil.csv"}\NormalTok{, }\StringTok{";"}\NormalTok{, }\DataTypeTok{escape_double =} \OtherTok{FALSE}\NormalTok{, }\DataTypeTok{trim_ws =} \OtherTok{TRUE}\NormalTok{)}
\end{Highlighting}
\end{Shaded}

\begin{verbatim}
## Parsed with column specification:
## cols(
##   DATA_INICIAL = col_character(),
##   DATA_FINAL = col_character(),
##   PRODUTO = col_character(),
##   POSTOS_PESQUISADOS = col_integer(),
##   UNIDADE_MEDIDA = col_character(),
##   PRECO_MEDIO_VENDA = col_number(),
##   DESVIO_PADRAO_REVENDA = col_character(),
##   PRECO_MINIMO_REVENDA = col_number(),
##   PRECO_MAXIMO_REVENDA = col_number(),
##   MARGEM_MEDIA_REVENDA = col_character(),
##   COEF_DE_VARIACAO_REVENDA = col_character(),
##   PRECO_MEDIO_DISTRIBUICAO = col_number(),
##   DESVIO_PADRAO_DISTRIBUICAO = col_character(),
##   PRECO_MINIMO_DISTRIBUICAO = col_character(),
##   PRECO_MAXIMO_DISTRIBUICAO = col_number(),
##   COEF_DE_VARIACAO_DISTRIBUICAO = col_character()
## )
\end{verbatim}

Imprimindo na tela as primeiras 6 linhas.

\begin{Shaded}
\begin{Highlighting}[]
\KeywordTok{head}\NormalTok{(anpbrasil)}
\end{Highlighting}
\end{Shaded}

\begin{verbatim}
## # A tibble: 6 x 16
##   DATA_INICIAL DATA_FINAL PRODUTO POSTOS_PESQUISA~ UNIDADE_MEDIDA
##   <chr>        <chr>      <chr>              <int> <chr>         
## 1 04/11/2012   10/11/2012 ETANOL~             8195 R$/l          
## 2 11/11/2012   17/11/2012 ETANOL~             8184 R$/l          
## 3 18/11/2012   24/11/2012 ETANOL~             8176 R$/l          
## 4 25/11/2012   01/12/2012 ETANOL~             8164 R$/l          
## 5 02/12/2012   08/12/2012 ETANOL~             8168 R$/l          
## 6 09/12/2012   15/12/2012 ETANOL~             8155 R$/l          
## # ... with 11 more variables: PRECO_MEDIO_VENDA <dbl>,
## #   DESVIO_PADRAO_REVENDA <chr>, PRECO_MINIMO_REVENDA <dbl>,
## #   PRECO_MAXIMO_REVENDA <dbl>, MARGEM_MEDIA_REVENDA <chr>,
## #   COEF_DE_VARIACAO_REVENDA <chr>, PRECO_MEDIO_DISTRIBUICAO <dbl>,
## #   DESVIO_PADRAO_DISTRIBUICAO <chr>, PRECO_MINIMO_DISTRIBUICAO <chr>,
## #   PRECO_MAXIMO_DISTRIBUICAO <dbl>, COEF_DE_VARIACAO_DISTRIBUICAO <chr>
\end{verbatim}

2.Mostre a que classe pertence cada uma das 10 colunas do seu
\texttt{dataframe}

\begin{verbatim}
## [1] "DATA_INICIAL =  character"
\end{verbatim}

\begin{verbatim}
## [1] "DATA_INICIAL =  character"
\end{verbatim}

\begin{verbatim}
## [1] "DATA_FINAL =  character"
\end{verbatim}

\begin{verbatim}
## [1] "PRODUTO =  character"
\end{verbatim}

\begin{verbatim}
## [1] "POSTOS_PESQUISADOS =  integer"
\end{verbatim}

\begin{verbatim}
## [1] "UNIDADE_MEDIDA =  character"
\end{verbatim}

\begin{verbatim}
## [1] "PRECO_MEDIO_VENDA =  numeric"
\end{verbatim}

\begin{verbatim}
## [1] "DESVIO_PADRAO_REVENDA =  character"
\end{verbatim}

\begin{verbatim}
## [1] "PRECO_MINIMO_REVENDA =  numeric"
\end{verbatim}

\begin{verbatim}
## [1] "PRECO_MAXIMO_REVENDA\t=  numeric"
\end{verbatim}

\begin{verbatim}
## [1] "MARGEM_MEDIA_REVENDA\t=  character"
\end{verbatim}

\begin{verbatim}
## [1] "COEF_DE_VARIACAO_REVENDA =  character"
\end{verbatim}

\begin{verbatim}
## [1] "PRECO_MEDIO_DISTRIBUICAO\t=  numeric"
\end{verbatim}

\begin{verbatim}
## [1] "DESVIO_PADRAO_DISTRIBUICAO =  character"
\end{verbatim}

\begin{verbatim}
## [1] "PRECO_MINIMO_DISTRIBUICAO =  character"
\end{verbatim}

\begin{verbatim}
## [1] "PRECO_MAXIMO_DISTRIBUICAO =  numeric"
\end{verbatim}

\begin{verbatim}
## [1] "COEF_DE_VARIACAO_DISTRIBUICAO =  character"
\end{verbatim}

3.Utilize um comando que faça uma avaliação exploratória das variáveis
do seu dataframe e imprima na tela.

\begin{Shaded}
\begin{Highlighting}[]
\KeywordTok{summary}\NormalTok{(anpbrasil)}
\end{Highlighting}
\end{Shaded}

\begin{verbatim}
##  DATA_INICIAL        DATA_FINAL          PRODUTO         
##  Length:40          Length:40          Length:40         
##  Class :character   Class :character   Class :character  
##  Mode  :character   Mode  :character   Mode  :character  
##                                                          
##                                                          
##                                                          
##  POSTOS_PESQUISADOS UNIDADE_MEDIDA     PRECO_MEDIO_VENDA
##  Min.   : 491       Length:40          Min.   :   40    
##  1st Qu.:7078       Class :character   1st Qu.: 1758    
##  Median :8052       Mode  :character   Median : 2146    
##  Mean   :6499                          Mean   : 8612    
##  3rd Qu.:8181                          3rd Qu.: 2749    
##  Max.   :8685                          Max.   :40151    
##  DESVIO_PADRAO_REVENDA PRECO_MINIMO_REVENDA PRECO_MAXIMO_REVENDA
##  Length:40             Min.   :  18         Min.   :  62        
##  Class :character      1st Qu.:1021         1st Qu.: 295        
##  Mode  :character      Median :1399         Median : 312        
##                        Mean   :1298         Mean   :1378        
##                        3rd Qu.:1869         3rd Qu.:2449        
##                        Max.   :2279         Max.   :3799        
##  MARGEM_MEDIA_REVENDA COEF_DE_VARIACAO_REVENDA PRECO_MEDIO_DISTRIBUICAO
##  Length:40            Length:40                Min.   :  157           
##  Class :character     Class :character         1st Qu.: 1599           
##  Mode  :character     Mode  :character         Median : 1927           
##                                                Mean   : 7366           
##                                                3rd Qu.: 2365           
##                                                Max.   :29857           
##  DESVIO_PADRAO_DISTRIBUICAO PRECO_MINIMO_DISTRIBUICAO
##  Length:40                  Length:40                
##  Class :character           Class :character         
##  Mode  :character           Mode  :character         
##                                                      
##                                                      
##                                                      
##  PRECO_MAXIMO_DISTRIBUICAO COEF_DE_VARIACAO_DISTRIBUICAO
##  Min.   :  44.0            Length:40                    
##  1st Qu.: 242.0            Class :character             
##  Median : 289.0            Mode  :character             
##  Mean   : 938.5                                         
##  3rd Qu.:1846.0                                         
##  Max.   :2927.0
\end{verbatim}

4.Adicione mais uma coluna gerada a partir da transformação de uma das
10 originais.

\begin{Shaded}
\begin{Highlighting}[]
\NormalTok{anpbrasil}\OperatorTok{$}\NormalTok{TOTAL_POSTOS<-(anpbrasil}\OperatorTok{$}\NormalTok{POSTOS_PESQUISADOS}\OperatorTok{/}\KeywordTok{sum}\NormalTok{(anpbrasil}\OperatorTok{$}\NormalTok{POSTOS_PESQUISADOS))}\OperatorTok{/}\DecValTok{100}
\end{Highlighting}
\end{Shaded}

5.Faça um gráfico que mostre a distribuição dos dados dessa nova
variável do exercício 4.


\end{document}
