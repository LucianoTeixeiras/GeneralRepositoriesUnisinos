\documentclass[]{article}
\usepackage{lmodern}
\usepackage{amssymb,amsmath}
\usepackage{ifxetex,ifluatex}
\usepackage{fixltx2e} % provides \textsubscript
\ifnum 0\ifxetex 1\fi\ifluatex 1\fi=0 % if pdftex
  \usepackage[T1]{fontenc}
  \usepackage[utf8]{inputenc}
\else % if luatex or xelatex
  \ifxetex
    \usepackage{mathspec}
  \else
    \usepackage{fontspec}
  \fi
  \defaultfontfeatures{Ligatures=TeX,Scale=MatchLowercase}
\fi
% use upquote if available, for straight quotes in verbatim environments
\IfFileExists{upquote.sty}{\usepackage{upquote}}{}
% use microtype if available
\IfFileExists{microtype.sty}{%
\usepackage{microtype}
\UseMicrotypeSet[protrusion]{basicmath} % disable protrusion for tt fonts
}{}
\usepackage[margin=1in]{geometry}
\usepackage{hyperref}
\hypersetup{unicode=true,
            pdftitle={A Renda per capita de Municípios Gaúchos},
            pdfauthor={Luciano Teixeira},
            pdfborder={0 0 0},
            breaklinks=true}
\urlstyle{same}  % don't use monospace font for urls
\usepackage{color}
\usepackage{fancyvrb}
\newcommand{\VerbBar}{|}
\newcommand{\VERB}{\Verb[commandchars=\\\{\}]}
\DefineVerbatimEnvironment{Highlighting}{Verbatim}{commandchars=\\\{\}}
% Add ',fontsize=\small' for more characters per line
\usepackage{framed}
\definecolor{shadecolor}{RGB}{248,248,248}
\newenvironment{Shaded}{\begin{snugshade}}{\end{snugshade}}
\newcommand{\KeywordTok}[1]{\textcolor[rgb]{0.13,0.29,0.53}{\textbf{#1}}}
\newcommand{\DataTypeTok}[1]{\textcolor[rgb]{0.13,0.29,0.53}{#1}}
\newcommand{\DecValTok}[1]{\textcolor[rgb]{0.00,0.00,0.81}{#1}}
\newcommand{\BaseNTok}[1]{\textcolor[rgb]{0.00,0.00,0.81}{#1}}
\newcommand{\FloatTok}[1]{\textcolor[rgb]{0.00,0.00,0.81}{#1}}
\newcommand{\ConstantTok}[1]{\textcolor[rgb]{0.00,0.00,0.00}{#1}}
\newcommand{\CharTok}[1]{\textcolor[rgb]{0.31,0.60,0.02}{#1}}
\newcommand{\SpecialCharTok}[1]{\textcolor[rgb]{0.00,0.00,0.00}{#1}}
\newcommand{\StringTok}[1]{\textcolor[rgb]{0.31,0.60,0.02}{#1}}
\newcommand{\VerbatimStringTok}[1]{\textcolor[rgb]{0.31,0.60,0.02}{#1}}
\newcommand{\SpecialStringTok}[1]{\textcolor[rgb]{0.31,0.60,0.02}{#1}}
\newcommand{\ImportTok}[1]{#1}
\newcommand{\CommentTok}[1]{\textcolor[rgb]{0.56,0.35,0.01}{\textit{#1}}}
\newcommand{\DocumentationTok}[1]{\textcolor[rgb]{0.56,0.35,0.01}{\textbf{\textit{#1}}}}
\newcommand{\AnnotationTok}[1]{\textcolor[rgb]{0.56,0.35,0.01}{\textbf{\textit{#1}}}}
\newcommand{\CommentVarTok}[1]{\textcolor[rgb]{0.56,0.35,0.01}{\textbf{\textit{#1}}}}
\newcommand{\OtherTok}[1]{\textcolor[rgb]{0.56,0.35,0.01}{#1}}
\newcommand{\FunctionTok}[1]{\textcolor[rgb]{0.00,0.00,0.00}{#1}}
\newcommand{\VariableTok}[1]{\textcolor[rgb]{0.00,0.00,0.00}{#1}}
\newcommand{\ControlFlowTok}[1]{\textcolor[rgb]{0.13,0.29,0.53}{\textbf{#1}}}
\newcommand{\OperatorTok}[1]{\textcolor[rgb]{0.81,0.36,0.00}{\textbf{#1}}}
\newcommand{\BuiltInTok}[1]{#1}
\newcommand{\ExtensionTok}[1]{#1}
\newcommand{\PreprocessorTok}[1]{\textcolor[rgb]{0.56,0.35,0.01}{\textit{#1}}}
\newcommand{\AttributeTok}[1]{\textcolor[rgb]{0.77,0.63,0.00}{#1}}
\newcommand{\RegionMarkerTok}[1]{#1}
\newcommand{\InformationTok}[1]{\textcolor[rgb]{0.56,0.35,0.01}{\textbf{\textit{#1}}}}
\newcommand{\WarningTok}[1]{\textcolor[rgb]{0.56,0.35,0.01}{\textbf{\textit{#1}}}}
\newcommand{\AlertTok}[1]{\textcolor[rgb]{0.94,0.16,0.16}{#1}}
\newcommand{\ErrorTok}[1]{\textcolor[rgb]{0.64,0.00,0.00}{\textbf{#1}}}
\newcommand{\NormalTok}[1]{#1}
\usepackage{graphicx,grffile}
\makeatletter
\def\maxwidth{\ifdim\Gin@nat@width>\linewidth\linewidth\else\Gin@nat@width\fi}
\def\maxheight{\ifdim\Gin@nat@height>\textheight\textheight\else\Gin@nat@height\fi}
\makeatother
% Scale images if necessary, so that they will not overflow the page
% margins by default, and it is still possible to overwrite the defaults
% using explicit options in \includegraphics[width, height, ...]{}
\setkeys{Gin}{width=\maxwidth,height=\maxheight,keepaspectratio}
\IfFileExists{parskip.sty}{%
\usepackage{parskip}
}{% else
\setlength{\parindent}{0pt}
\setlength{\parskip}{6pt plus 2pt minus 1pt}
}
\setlength{\emergencystretch}{3em}  % prevent overfull lines
\providecommand{\tightlist}{%
  \setlength{\itemsep}{0pt}\setlength{\parskip}{0pt}}
\setcounter{secnumdepth}{5}
% Redefines (sub)paragraphs to behave more like sections
\ifx\paragraph\undefined\else
\let\oldparagraph\paragraph
\renewcommand{\paragraph}[1]{\oldparagraph{#1}\mbox{}}
\fi
\ifx\subparagraph\undefined\else
\let\oldsubparagraph\subparagraph
\renewcommand{\subparagraph}[1]{\oldsubparagraph{#1}\mbox{}}
\fi

%%% Use protect on footnotes to avoid problems with footnotes in titles
\let\rmarkdownfootnote\footnote%
\def\footnote{\protect\rmarkdownfootnote}

%%% Change title format to be more compact
\usepackage{titling}

% Create subtitle command for use in maketitle
\newcommand{\subtitle}[1]{
  \posttitle{
    \begin{center}\large#1\end{center}
    }
}

\setlength{\droptitle}{-2em}

  \title{A Renda per capita de Municípios Gaúchos}
    \pretitle{\vspace{\droptitle}\centering\huge}
  \posttitle{\par}
  \subtitle{Região Metropolitana de Porto Alegre}
  \author{Luciano Teixeira}
    \preauthor{\centering\large\emph}
  \postauthor{\par}
      \predate{\centering\large\emph}
  \postdate{\par}
    \date{30 de junho de 2018}


\begin{document}
\maketitle

{
\setcounter{tocdepth}{2}
\tableofcontents
}
\section{Introdução da Análise}\label{introducao-da-analise}

O arquivo utilizado, se refere aos dados municipais do Atlas do
desenvolvimento humano no Brasil referentes aos Censos de 1991, 2000 e
2010 em \url{http://www.atlasbrasil.org.br/2013/pt/download/}.

Foram escolhidas 5 variáveis explicativas para a renda per capita dos
municípios.

\begin{itemize}
\tightlist
\item
  IDHM: Índice de Desenvolvimento Humano Municipal
\item
  ESPVIDA: Esperança de vida ao nascer
\item
  GINI: Índice de Gini
\item
  PESOURB: População residente na área urbana
\item
  T\_FBSUPER: Taxa de frequência bruta ao ensino superior
\end{itemize}

A amostra será demonstrada por meio de uma análise descritiva des
variáveis explicaivas em relação à evolução da renda per capita dos
municípios da região metropolitana de Porto Alegre sobre os anos de
1991, 2000 e 2010.

Como método de análise, será utilizado regressão linear múltipla onde a
VR é a renda per capita e as variáveis explicativas são as 5 escolhidas
no passo 2.

\section{Inicializando Bibliotecas}\label{inicializando-bibliotecas}

Como primeiro passo, serão carregadas a seguintes bibliotecas. Caso
estas não se encontrem instaladas, é necessário que esta instalação seja
eetuada.

\begin{Shaded}
\begin{Highlighting}[]
\KeywordTok{library}\NormalTok{(readr)}
\KeywordTok{library}\NormalTok{(dplyr)}
\end{Highlighting}
\end{Shaded}

\begin{verbatim}
## 
## Attaching package: 'dplyr'
\end{verbatim}

\begin{verbatim}
## The following objects are masked from 'package:stats':
## 
##     filter, lag
\end{verbatim}

\begin{verbatim}
## The following objects are masked from 'package:base':
## 
##     intersect, setdiff, setequal, union
\end{verbatim}

\begin{Shaded}
\begin{Highlighting}[]
\KeywordTok{library}\NormalTok{(readxl)}
\KeywordTok{library}\NormalTok{(ggplot2)}
\KeywordTok{library}\NormalTok{(stringi)}
\KeywordTok{library}\NormalTok{(stringr)}
\end{Highlighting}
\end{Shaded}

\section{Importando Dados Brutos}\label{importando-dados-brutos}

\begin{Shaded}
\begin{Highlighting}[]
\NormalTok{dadosbrutos <-}\StringTok{ }\KeywordTok{read_excel}\NormalTok{(}\StringTok{"atlas2013_municipios.xlsx"}\NormalTok{)}
\end{Highlighting}
\end{Shaded}

\section{Especificando os Dados}\label{especificando-os-dados}

Comandos Encadeados podem demonstrar um principio de Machine Learning,
segregando cidades, Estado e Região. No caso deste modelo, oi delimitado
a Região Metropolitana de Porto Alegre, podendo ser aplicado em qualquer
estado, macro região ou micro região, com pequenos ajustes.

Este encadeameno de funçoes, substiui uma série de passos, utilizados
anteriormente para chegar à um resultado muito mais enchuto, levando em
consideração proficionais de analise de dados com poucos recursos em
questão de equipamenos, como por exemplos computadores de pequeno porte,
pouca memória e processador limitado.

\begin{Shaded}
\begin{Highlighting}[]
\NormalTok{dadosrs <-}
\StringTok{  }\KeywordTok{filter}\NormalTok{(}
    \KeywordTok{select}\NormalTok{(}
      \KeywordTok{subset.data.frame}\NormalTok{(dadosbrutos, UF }\OperatorTok{==}\StringTok{ }\DecValTok{43}\NormalTok{),}
\NormalTok{      ANO,}
\NormalTok{      UF,}
\NormalTok{      MUNICIPIO,}
\NormalTok{      RDPC,}
\NormalTok{      IDHM,}
\NormalTok{      ESPVIDA,}
\NormalTok{      GINI,}
\NormalTok{      PESOURB,}
\NormalTok{      T_FBSUPER}
\NormalTok{    ),}
\NormalTok{    MUNICIPIO }\OperatorTok\StringTok{ }\KeywordTok{c}\NormalTok{(}\StringTok{"NOVO HAMBURGO"}\NormalTok{, }\StringTok{"SÃO LEOPOLDO"}\NormalTok{, }\StringTok{"SAPUCAIA DO SUL"}\NormalTok{,}
                     \StringTok{"ESTEIO"}\NormalTok{, }\StringTok{"CANOAS"}\NormalTok{, }\StringTok{"PORTO ALEGRE"}\NormalTok{, }\StringTok{"GUAÍBA"}\NormalTok{)}
\NormalTok{  )}
\end{Highlighting}
\end{Shaded}

\section{Listando os Dados}\label{listando-os-dados}

\begin{Shaded}
\begin{Highlighting}[]
\KeywordTok{head}\NormalTok{(dadosrs)}
\end{Highlighting}
\end{Shaded}

\begin{verbatim}
## # A tibble: 6 x 9
##     ANO    UF MUNICIPIO      RDPC  IDHM ESPVIDA  GINI PESOURB T_FBSUPER
##   <dbl> <dbl> <chr>         <dbl> <dbl>   <dbl> <dbl>   <dbl>     <dbl>
## 1  1991    43 CANOAS         522. 0.556    69.0 0.5    269258     12.5 
## 2  1991    43 ESTEIO         535. 0.589    69.5 0.48    70468     15.8 
## 3  1991    43 GUAÍBA         402. 0.522    70.0 0.48    72731      8.3 
## 4  1991    43 NOVO HAMBURGO  614. 0.544    68.9 0.53   201502      8.63
## 5  1991    43 PORTO ALEGRE  1022. 0.66     69.9 0.570 1236024     32.9 
## 6  1991    43 SÃO LEOPOLDO   565. 0.543    67.9 0.55   160358     17.8
\end{verbatim}

Total de 21 registros.


\end{document}
